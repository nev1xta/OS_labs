\section{Метод решения}
Данная программа реализует многопроцессную обработку текстовых данных с использованием каналов (pipes) для межпроцессного взаимодействия. Родительский процесс читает строки из стандартного ввода и направляет их через цепочку дочерних процессов, каждый из которых выполняет преобразование данных.

{\bfseries Основные компоненты:}

Parent - управляет созданием каналов, запуском дочерних процессов, принимает пользовательский ввод и выводит конечный результат;

Child1 - приводит текст к нижнему регистру;

Child2 - удаляет все задвоенные пробелы;

Разделение системных вызовов в отдельную библиотеку systemCall.

\section{Описание программы}

{\bfseries Структура проекта:}

lab1/

\hspace{3em}report/
    
\hspace{6em}...
        
\hspace{3em}include/
    
\hspace{6em}systemCall.h    // Заголовочный файл библиотеки
         
\hspace{3em}src/
    
\hspace{6em}systemCall.cpp  // Реализация системных функций
        
\hspace{6em}parent.cpp         // Родительский процесс
        
\hspace{6em}child1.cpp         // Дочерний процесс 1 (нижний регистр)
        
\hspace{6em}child2.cpp         // Дочерний процесс 2 (удаление пробелов)
        
\hspace{3em}CMakeLists.txt

{\bfseries Основные типы данных:}

1.Структура pipeT (канал)

Содержит два конца: один для чтения данных, другой для записи

На Windows использует дескрипторы HANDLE, на Linux - файловые дескрипторы

Позволяет организовать однонаправленную передачу данных между процессами

2.Структура process (информация о процессе)

Хранит идентификатор запущенного процесса

На Windows содержит подробную информацию о процессе, на Linux - просто номер процесса (PID)

Содержит флаг is-valid, который показывает, работает ли процесс корректно

3.Строки std::string

4.Логические флаги (bool)

{\bfseries Принцип работы с типами данных:}

Программа создает несколько каналов pipe t, через которые передаются строки std::string. Каждый дочерний процесс управляется через свою структуру process info t, а логические флаги следят за тем, чтобы вся система работала без ошибок.

{\bfseries Основные функции программы:}

PipeCreate() - создает новый канал для передачи данных

PipeClose() - полностью закрывает канал, освобождая ресурсы

ProcessCreate() - запускает дочерний процесс

ProcessTerminate() - принудительно завершает процесс

ReadStringFromPipe() - читает строку из канала

WriteStringToPipe() - записывает строку в канал

{\bfseries Используемые системные вызовы:}

Для Windows:

CreatePipe() - создание канала;

CreateProcess() - создание процесса;

CloseHandle() - закрытие дескриптора;

ReadFile()/WriteFile() - работа с каналами;

TerminateProcess() - принудительное завершение;

Для Linux:

pipe() - создание канала;

fork() - создание процесса;

exec() - загрузка новой программы;

close() - закрытие дескриптора;

read()/write() - работа с каналами;

kill() - отправка сигнала процессу;

dup2() - перенаправление стандартных потоков.
